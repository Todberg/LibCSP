\section{Compliance Levels}
\label{section:complianceLevels}
Although the Safety-Critical Java specification was developed with simplification in mind, the complexity across different applications can vary greatly. One application may for example contain a single periodic event handler performing some trivial operation whereas another application may contain multiple periodic and aperiodic event handlers executing under strict timing constraints, performing complex operations and with synchronisation in between. In order to minimise the certification process, the specification was created in a way such that simple applications do not have to suffer with a resource heavy infrastructure and an equally heavy Java run-time environment each causing extra certification effort. The financial cost of certifying a SCJ application, and any safety-critical application for that matter, namely depends on its complexity. The more complex the application is, the more expensive and longer it takes to certify it. The Safety-Critical Java specification defines three levels for this purpose that are referred to as, \textit{compliance levels}. The levels are denoted \textit{level 0}, \textit{level 1} and \textit{level 3} respectively where simple applications may conform to level 0 and complex ones to level 3. It is thus desirable to minimise the compliance level as much as possible. It should be noted that compliance levels are \textit{not} related to the criticality levels from the previously mentioned DO-178B standard.