\section{Input / Output}
\label{section:inputOutput}
An integral part of safety-critical systems (and inherently with often being an embedded system) is the need for communicating with external devices and doing basic I/O. This typically involves accessing control, status and data registers of an hardware device and handling interrupts generated by such external devices.

SCJ uses the \textit{Raw Memory Access} approach from RTSJ to access physical memory addresses using primitive types in Java. This allows SCJ programs to access memory mapped I/O or DMA enabled devices. The actual use of \code{RawMemory} objects and especially instantiation of these are implementation defined. The specification declares that a factory will be available that provides the application developer the means to access the raw memory that the underlying infrastructure supports.

In addition to accessing raw memory, SCJ specifies a model for interrupts that allows the application developer to provide interrupt handlers. This is done by implementing the ISR in a class that inherits from \code{ManagedInterruptServiceRoutine}. Because the use of I/O is much dependant on the implementation we refer to the specification for additional details on raw memory access and interrupt handling.