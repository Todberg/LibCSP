%This project documents the development of a smartphone application for the Android platform. The application, UniNav, is an indoor navigation mobile application primarily designed for the students at the University of Aalborg and the layout of the Cassiopeia building to help them in checking their agenda for the lectures they are subscribed to, and find the direction where a specific lecture is held or a specific point of interest within the building. It also helps the different visitors of the University in finding of some specific point of interests inside the building.\\

%The development of the implemented application has been done by using Android platform with eclipse IDE as programming environment and Java as programming language. As development methodology, an   agile development method, scrum supported by pair programming, has been slightly adapted to accommodate the project environment. The application provides the map services functionalities implemented by using SmartCampusAAU library, which used Wi-Fi fingerprinting technologies. The scheduling information will come from a remote web application i.e. Moodle in our case, fetched into the Android application, which has an integrated database management system, SQLite which will handle these data.\\ A usability evaluation has been conducted to verify that the developed application meets the needs of the user. The analysis of the results from the usability evaluation helped us to strengthen the performances of the UniNav. The perspectives and suggestions on the developed application, ending by an overall conclusion have been discussed.

This report documents the development of UniNav, a smartphone application primarily designed to help university students in keeping up to date with their agenda, as well as providing indoor navigation for locating lectures and other points of interests. It also features a visitor mode, helping visitors find points of interests inside buildings.

The development consists of two parts, a mobile app for the Android platform, and a web service, with most focus on the mobile application. Currently, the app is designed to work with the Cassiopeia building of Aalborg University, and the software engineer study. The web service automatically handles fetching calendar updates from the internal system, Moodle, and the app is used for presenting this and providing indoor navigation to points of interests, for example where a lecture is held. The report also documents how we used the Scrum development methodology for conducting the development. 